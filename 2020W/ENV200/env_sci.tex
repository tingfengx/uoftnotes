\documentclass[10pt]{article}
\usepackage{xeCJK}
\usepackage{geometry}
\usepackage[utf8]{inputenc}
\usepackage[english]{babel}
\usepackage{amsmath}
\usepackage{amssymb}
\usepackage{amsfonts}
\usepackage[dvipsnames]{xcolor}
\usepackage{hyperref}
\usepackage{fancyhdr}
\usepackage{datetime}
\usepackage{ccicons}
\usepackage{mdframed}
%\usepackage[outputdir=.texpadtmp]{minted}

% ==== License =====
\usepackage[
    type={CC}, 
    modifier={by-nc-sa}, 
    version={4.0},
]{doclicense}

% General
\newcommand{\mc}[1]{\mathcal{#1}}

% Math Bold Font, Vector Notations
\newcommand{\ba}{\mathbf{a}}
\newcommand{\bb}{\mathbf{b}}
\newcommand{\bc}{\mathbf{c}}
\newcommand{\bd}{\mathbf{d}}
\newcommand{\be}{\mathbf{e}}
\renewcommand{\bf}{\mathbf{f}}
\newcommand{\bg}{\mathbf{g}}
\newcommand{\bh}{\mathbf{h}}
\newcommand{\bi}{\mathbf{i}}
\newcommand{\bj}{\mathbf{j}}
\newcommand{\bk}{\mathbf{k}}
\newcommand{\bl}{\mathbf{l}}
\newcommand{\bm}{\mathbf{m}}
\newcommand{\bn}{\mathbf{n}}
\newcommand{\bo}{\mathbf{o}}
\newcommand{\bp}{\mathbf{p}}
\newcommand{\bq}{\mathbf{q}}
\newcommand{\br}{\mathbf{r}}
\newcommand{\bs}{\mathbf{s}}
\newcommand{\bt}{\mathbf{t}}
\newcommand{\bu}{\mathbf{u}}
\newcommand{\bv}{\mathbf{v}}
\newcommand{\bw}{\mathbf{w}}
\newcommand{\bx}{\mathbf{x}}
\newcommand{\by}{\mathbf{y}}
\newcommand{\bz}{\mathbf{z}}
\newcommand{\bzero}{\mathbf{0}}

% Proofs, Structures
\newcommand{\proof}{\tit{\underline{Proof:}}} % This equivalent to the \begin{proof}\end{proof} block
\newcommand{\proofforward}{\tit{\underline{Proof($\implies$):}}}
\newcommand{\proofback}{\tit{\underline{Proof($\impliedby$):}}}
\newcommand{\proofsuperset}{\tit{\underline{Proof($\supseteq$):}}}
\newcommand{\proofsubset}{\tit{\underline{Proof($\subseteq$):}}}
\newcommand{\contradiction}{$\longrightarrow\!\longleftarrow$}
\newcommand{\qed}{\hfill $\mathcal{Q}.\mathcal{E}.\mathcal{D}.\dagger$}

% Number Spaces, Vector Space
\newcommand{\R}{\mathbb{R}}
\newcommand{\real}{\mathbb{R}}
\newcommand{\complex}{\mathbb{C}}
\newcommand{\field}{\mathbb{F}}

% customized commands
\newcommand{\settag}[1]{\renewcommand{\theenumi}{#1}}
\newcommand{\tbf}[1]{\textbf{#1}}
\newcommand{\tit}[1]{\textit{#1}}
\newcommand{\overbar}[1]{\mkern 1.5mu\overline{\mkern-1.5mu#1\mkern-1.5mu}\mkern 1.5mu}
\newcommand{\double}[1]{\mathbb{#1}} % Set to behave like that on word
\newcommand{\trans}[3]{$#1:#2\rightarrow{}#3$}
\newcommand{\map}[3]{\text{$\left[#1\right]_{#2}^{#3}$}}
\newcommand{\dime}[1]{\text{dim}(#1)}
\newcommand{\mat}[2]{M_{#1 \times #2}(\R)}
\newcommand{\aug}{\fboxsep=-\fboxrule\!\!\!\fbox{\strut}\!\!\!}
\newcommand{\basecase}{\textsc{\underline{Basis Case:}} }
\newcommand{\inductive}{\textsc{\underline{Inductive Step:}} }
\newcommand{\norm}[1]{\left\lVert#1\right\rVert}
\newcommand{\independent}{\perp \!\!\! \perp}

\author{\ccLogo \,\,by Xia, Tingfeng}
\title{\textsc{Science and Environment}}
\date{\today}

\begin{document}
\maketitle
\doclicenseThis
\section*{Preface}
\begin{quote}
    Earth’s natural system is undergoing considerable change. Although these changes are following a  natural  cyclical  path, the  past  200  years  have  seen  an  accelerated rate,  scale  and  scope  of change not  witnessed  before.  To  understand  and  assess  the  global  impacts  of  these  changes, ENV200  has  been  designed  to examine  them through a  scientific lens. These impacts  have global implications: atmospheric systems and climate change, the biosphere and conservation of biodiversity.
\end{quote}
\hfill --- Syllabus

\tableofcontents
\newpage
\section{Lecture 1 - Understanding Environmental Issues and Science}
\subsection{Learning Outcomes}
\begin{itemize}
    \item Describe several important environmental problems facing the world.
    \begin{mdframed}
        There are seven major problems discussed in this lecture, they are climate change, hunger, clean water, energy resources, air quality, biodiversity loss, and marine resources. 
    \end{mdframed}
    \item List several examples of progress in environmental quality.
    \begin{mdframed}
        Plateau of population growth, decrease in life-threatening diseases, more access-able renewable energy source, more access-able education due to Internet, decrease in rate of deforestation, and better protected marine resources. 
    \end{mdframed}
    \item Explain the idea of sustainability and some of its aims.
    \begin{mdframed}
        Sustainability is a search for long term ecological stability and human progress. Quote: ``meeting the needs of the present without compromising the ability of future generations to meet their own needs''
    \end{mdframed}
\end{itemize}

\subsection{Definition - Environmental Science}
\begin{itemize}
    \item Environmental science is the systematic study of our environment and out place in it, and
    \item Environmental science draws on many fields of knowledge to fully understand a problem and solve it. 
\end{itemize}

\subsection{Environmental and Political Problems}
\subsubsection{Climate Change} Human activities have greatly increased concentrations of carbon dioxide and other greenhouse gases over the last 200 years. Climate models indicate that by 2100, if current trends continue, global mean temperatures will probably warm between about 2 and 6 degrees Celsius. 

\subsubsection{Hunger} Over the past century, global food production has increased faster than human population growth, but hunger remains a chronic problem. At least 60 million people face acute food shortages due to weather, politics, or war. 

\subsubsection{Clean Water}
\begin{itemize}
    \item 1.1 billion people lack access to safe drinking water.
    \item Every year polluted water contributes to the death of more than 15 million people.
    \item 40 \% of the population live in countries where water demands now exceed supplies. 
\end{itemize}

\subsubsection{Energy Resources}
\begin{itemize}
    \item Fossil fuels (oil, coal, and natural gas) presently provide around 80 percent of the energy used in industrialized countries.
    \item Supplies of these fuels are diminishing, and there are many problems associated with their acquisition and use.
    \item Investing in renewable energy and energy conservation measures could give us cleaner, less destructive options.
\end{itemize}

\subsubsection{Air Quality}
\begin{itemize}
    \item Air quality has worsened dramatically in many areas, especially China and India.
    \item Nobel laureate Paul Crutzen estimates that at least 3 million people die each year from diseases triggered by air pollution.
    \item Word-wide, the UN estimates that more than 2 billion metric tons of air pollutants (\textit{which doesn't include carbon dioxide, or wind-blown soil}) are released each year.
\end{itemize}

\subsubsection{Biodiversity Loss}
Habitat destruction, overexploitation, pollution, and introduction of exotic organisms are eliminating species at a rate comparable to the great extinction that marked the end of the age of dinosaurs.

\subsubsection{Marine Resources}
\begin{itemize}
    \item More than a billion people depend on seafood for their main source of animal protein.
    \item According to the World Resources Institute (WRI), more than three-quarters of the 441 fish stocks for which information is available are severely depleted or in urgent need of better management. 
\end{itemize}

\subsection{Signs of Progress}
\subsubsection{Population}
Population has stabilized in most industrialized countries where democracy has been established. 
\begin{itemize}
    \item Since 1960, the average number of children born per woman worldwide has decreased from 5.0 to 2.45. 
    \item The UN Population Division predicts that the world population will stabilize at about 8.9 billion by the year 2050.
\end{itemize}

\subsubsection{Health}
The incidence of life-threatening infectious diseases like smallpox and polio have been reduced sharply in most countries during the past century, while life expectancies have nearly doubled. 

\subsubsection{Renewable Energy}
Encouraging progress is being made in a transition to renewable energy sources.
\begin{itemize}
    \item Growth in wind energy, solar, and biomass power and improvements in efficiency are beginning to reduce reliance on fossil fuels.
    \item The cost of solar power has plummeted (dropped, at high speed), and both solar and wind power are now far cheaper, easier, and faster to install than nuclear power or new coal plants. 
\end{itemize}

\subsubsection{Information and Education}
\begin{itemize}
    \item Literacy and access to education are expanding in most regions of the world.
    \item The Internet makes it easier to share environmental solutions.
    \item Expanding education for girls is driving declining birth rates worldwide.
\end{itemize}

\subsubsection{Conservation of Forest and Nature Preserves}
\begin{itemize}
    \item Deforestation has slowed in Asia.
    \item A former leader in deforestation, Brazil, is now working to protect forests.
    \item 13.5\% of the world's land area is now in protected areas.
\end{itemize}

\subsubsection{Protection of Marine Resources}
\begin{itemize}
    \item Marine protected areas and better monitoring of provides for more sustainable management.
    \item Marine reserves have been established in California, Hawaii, New Zealand, Great Britain, and many other areas.
\end{itemize}

\subsection{Definition - Sustainability}
\begin{itemize}
    \item \textit{Sustainability} is a search for long term ecological stability and human progress.
    \item World Health Organization director Gro Harlem Brundtland has defined sustainable development as ``meeting the needs of the present without compromising the ability of future generations to meet their own needs.''
\end{itemize}

\subsection{Definition - Science}
\begin{itemize}
    \item Science is a process for producing knowledge based on observations. We develop or test theories (proposed explanations of how a process works) using these observations.
    \item Science also refers to the cumulative body of knowledge produced by many scientists.
    \item Science rests on the assumption that the world is knowable and that we can learn about it by careful observation and logical reasoning. 
\end{itemize}

\subsection{Orderly Procedure - Scientific Method}
\begin{enumerate}
    \item Make an observation and identify a question.
    \item Propose a hypothesis
    \item Test the hypothesis.
    \item Gather data from the test.
    \item Interpret the results. (Re-define and revise original hypothesis if it didn't work; Go back to step 2. )
    \item Report for peer review. 
\end{enumerate}

\subsection{Conclusions}
\begin{itemize}
    \item We face many persistent environmental problems, but we can also see many encouraging examples of progress.
    \item \color{Red} Resolving these multiple problems together is the challenge for sustainability.\color{Black}
    \item Science gives us an orderly, methodical approach to examining environmental problems. 
\end{itemize}


\section{Lecture 2 - Energy and Material Cycles}
\subsection{Energy}
\paragraph{(Definition) Energy} is the ability to do work, such as moving matter over a distance or causing a heat transfer between two objects at different temperatures. Energy can take many different forms, heat, light, electricity, and chemical energy are examples that we all experience. 

\paragraph{(Definition) kinetic Energy} is the energy contained in moving objects. Examples include a rock rolling down a hill, wind blowing through the trees, water flowing over a dam, or electrons speeding around the nucleus of an atom. 

\paragraph{(Definition) Potential Energy} is energy that are stored and is available to use. Examples include rick poised at the top of a hull and water stored behind a dam. \textbf{\textit{Note:}} Chemical energy stored in the food or gasoline are also forms of potential energy that can be released to do useful work. 

\paragraph{Basics of Energy}
\begin{itemize}
    \item Heat describes the energy that can be transferred between objects of different temperatures. 
    \item The study of thermodynamics deals with how energy is transferred in natural processes.
    \item The first law of thermodynamics states that energy is conserved
    \item The second law of thermodynamics states that, with each successive energy transfer or transformation in a system, less energy is available to do the work. (Energy transfer/transformation incurs a cost.)
    \item Entropy tends to increase in all natural systems. 
\end{itemize}

\paragraph{Life and Energy}
\begin{itemize}
    \item Nearly all life forms on earth relies on the sun for ultimate power source, either directly or indirectly.
    \item The energy is captured by green plates which are often called \textit{\textbf{primary producers}} because they create carbohydrates and other compounds using just sunlight, air, and water. 
    \item There are organisms that get energy in other ways, These organisms gain their energy from \textit{\textbf{chemosynthesis}}, a process which allows them to extract energy from inorganic chemical compounds such as hydrogen sulphide. 
\end{itemize}

\paragraph{Plants' Energy: Sun}
\begin{itemize}
    \item Thermonuclear reactions from our sun emit powerful forms of radiation, including potentially deadly ultraviolet and nuclear radiation. 
    \item Nearly all organisms on the earth's surface depend on solar radiation for life-sustaining energy, which is captured by green plants, algae, and some bacteria in a process called photosynthesis. 
    \item \textbf{(Definition) photosynthesis} converts light energy into useful, chemical energy in the bonds that hold together organic molecules. 
\end{itemize}


\subsection{Populations, Communities, and Ecosystems}
\paragraph{(Definition) Population} consists of all the members of a species living in a given area at the same time. 
\paragraph{(Definition) Community} All of the populations of organism living and interacting in particular area make up a community.
\paragraph{(Definition) Ecosystem} is composed of a biological community and its physical environment. 

\subsubsection{Food Chains, Food Webs, and Trophic Level Link Species}
\begin{itemize}
    \item In ecosystems, some consumers feed on a single species, but most consumers have multiple food sources. 
    \item Similarly, some species are prey to a single kind of predator, but many species in an ecosystem are beset by several types of predators and parasites. 
    \item In this way, individual food chains become interconnected to form a food web.
\end{itemize}

\paragraph{(Definition) Trophic Level}
\begin{itemize}
    \item A trophic level is an organism's feeding status in an ecosystem.
    \item Primary producers (or \textit{\textbf{autotrophs}}) feed themselves using only sunlight, water, carbon dioxide, and minerals. 
    \item Other organisms in the ecosystem are consumers (\textit{\textbf{or heterotrophs}}) of the chemical energy harnessed by the primary producers. 
    \item Herbivores are consumer who are eaters.
    \item Carnivores are flesh eaters. 
    \item Omnivores eat both plant and animal matter. 
\end{itemize}

\paragraph{Recyclers: Parasites, Scavengers, and Decomposers}
\begin{itemize}
    \item Like omnivores, these recyclers feed on all trophic levels. 
    \item Scavengers (e.g. jackals and vultures(秃)) clean up dead carcasses of larger animals. 
    \item Detritivores, such as ants and beetles, consume litter, debris, and dung. 
    \item Decomposer organisms, such as fungi and bacteria, complete the final breakdown and recycling of organic materials. 
\end{itemize}

%\subsection{Material Cycles}
%\subsubsection{Bio-Geo-Chemical Cycles}
\subsection{The Hydrologic Cycle}
\begin{itemize}
    \item The path of water through our environment is perhaps the most familiar bio-geo-chemical cycle. 
    \item Most of the earth's water is stored in the oceans, but solar energy continually evaporates this water, and wind distribute water vapour around the globe. 
    \item Water condenses over land surfaces, in the form of rain, snow, or fog, supporting terrestrial ecosystems, 
    \item Organisms emit the moisture they have consumed through respiration and perspiration. Eventually this moisture reenters the atmosphere or enters lakes and streams which ultimately returns it to the ocean. 
\end{itemize}

\section{Lecture3 - Environmental Systems}
\subsection{Bio-Geo-Chemical Cycles}
\begin{itemize}
    \item Nutrients are cycled continuously among different components of the ecosphere in characteristic paths known as biogeochemical cycles. 
    \item Generalized nutrient cycle model help represent the complexity of the earth's process.
    \item Under \textit{natural conditions}, recycling rates between components achieve a balance over time in which inputs and outputs are equal. 
    \item Human activities speed up transference between cycles components.
    \item Pollution problems result from human-induced accumulation in one or more components of a cycle. 
\end{itemize}

\subsubsection{The Carbon Cycle}
\begin{itemize}
    \item Carbon dioxide is the main reservoir for the carbon that is building block for all necessary fates, proteins, and carbohydrates that constitute life. 
    \item Plates take up carbon dioxide directly from the atmosphere through the process of photosynthesis, 
    \item Carbon is incorporated in biomass and passes along the food chain.
    \item Respiration by organism transforms some carbon in biomass back into carbon dioxide which enters atmosphere. 
    \item Cellular respiration by decomposers helps to return carbon from dead organisms into the atmosphere as carbon dioxide (methane-$CH_4$ in anaerobic conditions)
    \item The cycling of carbon and the flow of energy through food chains are intimately related.
    \item Carbon can be stored in the lithosphere (岩石圈) for extended periods of time as organisms become buried before they decompose. 
    \begin{itemize}
        \item This is particularly true under relatively inefficient anaerobic decay conditions such as in peat bogs (泥炭沼泽).
        \item Over millions of years, past forest, marine, and freshwater ecosystems have been transformed into fossil fuels through heat and compression. 
    \end{itemize}
\end{itemize}

\subsubsection{The Nitrogen Cycle}
%\paragraph{Nitrogen Facts}
\begin{itemize}
    \item Nitrogen is a colourless, tasteless, odourless gas required by all organisms for life. 
    \item 78\% of earth's atmosphere is composed of $N$. 
    \item \color{Red}The main way in which the atmospheric reservoir is lined to the biotic components of the food chain through \textit{\textbf{nitrogen fixation}}\footnote{TL;DR: Nitrogen fixation is a process by which molecular nitrogen in the air is converted into ammonia or related nitrogenous compounds in soil.} and \textit{\textbf{denitrification}}\footnote{The loss or removal of nitrogen or nitrogen compounds}\color{Black}
\end{itemize}
\paragraph{The Nitrogen Fixation Process}
\begin{itemize}
    \item \textit{\textbf{Nitrogen Fixation}} occurs as bacteria transform atmospheric nitrogen into various forms that are available to plant.
    \item The most important nitrogen fixers are bacteria of the Rhizobium family that grow on root nodules of plants.
    \item These bacteria combine gaseous $N_2$ with $H_2$ to make ammonia ($NH_3$) and ammonium ($NH_4^+$).
    \item Other bacteria then combine ammonia with oxygen to form nitrites ($NO_2^-$). 
    \item A third group of bacteria converts nitrites to nitrates, which green plants can absorb and use. 
    \item Plant cells absorb nitrates, and use them to build amino acids and eventually proteins. 
\end{itemize}


\paragraph{Mineralization} is the process by which decomposing biomass is converted back to ammonia and ammonium salts by bacterial action returned to the soil. 

\subsection{The Phosphorus Cycle}
\begin{itemize}
    \item The phosphorus cycle takes millions of years,
    \item Minerals become available to organisms after they are released from rocks or salts.
    \item Primary Producers take in inorganic phosphorus, incorporate it into organic molecules, and then pass it on to consumers. 
    \item In this way, phosphorus cycles through the ecosystem. 
    \item (富养化) Excess phosphates in bodies of water can stimulate explosive growth of algae, upsetting ecosystem stability.  
\end{itemize}

\subsection{The Sulphur Cycle}\footnote{Counter-intuitively enough, $S$ is actually a ``cooling gas''}
\begin{itemize}
    \item The form of $S$ in soils depends on the presence (aerobic) or absence (anaerobic) of oxygen (a reflection of the relationship of the particular site of transformation to the water table) and the presence of other elements such as iron. 
    \item Human activities also release large quantities of sulphur, primarily through burning fossil fuels. These precesses can contribute to acid precipitation.
    \item The biogenic sulphur emissions of oceanic phytoplankton may play a role in global climate
    \item The phytoplankton release sulphur compounds into the atmosphere which can reflect sunlight, cooling the earth. 
    \item This may be a feedback mechanism that keeps Earth's temperature within a suitable range for life. 
\end{itemize}



\section{Lecture4 - Biomes and Biodiversity}





















\end{document}
