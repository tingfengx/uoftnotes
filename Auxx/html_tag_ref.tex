% Meta data
\documentclass[oneside, 12pt]{article}
\usepackage[utf8x]{inputenc}
\usepackage[english]{babel}
\usepackage{amsfonts}
\usepackage{amsmath}
\usepackage{amssymb}
\usepackage{url}
\usepackage[pdfencoding=auto, psdextra]{hyperref}
\usepackage{fancyhdr} 
\usepackage{youngtab}

% customized commands
\newcommand{\settag}[1]{\renewcommand{\theenumi}{#1}}
\newcommand{\R}{\mathbb{R}}
\newcommand{\real}{\mathbb{R}}
\newcommand{\complex}{\mathbb{C}}
\newcommand{\field}{\mathbb{F}}
\newcommand{\double}[1]{\mathbb{#1}} % Set to behave like that on word
\newcommand{\qed}{\hfill $\mathcal{Q}.\mathcal{E}.\mathcal{D}.\dagger$}
\newcommand{\tbf}[1]{\textbf{#1}}
\newcommand{\tit}[1]{\textit{#1}}
\newcommand{\contradiction}{$\longrightarrow\!\longleftarrow$}
\newcommand{\overbar}[1]{\mkern 1.5mu\overline{\mkern-1.5mu#1\mkern-1.5mu}\mkern 1.5mu}
\newcommand{\proof}{\tit{\underline{Proof:}}} % This equivalent to the \begin{proof}\end{proof} block
\newcommand{\proofforward}{\tit{\underline{Proof($\itemmmplies$):}}}
\newcommand{\proofback}{\tit{\underline{Proof($\itemmmpliedby$):}}}
\newcommand{\proofsuperset}{\tit{\underline{Proof($\supseteq$):}}}
\newcommand{\proofsubset}{\tit{\underline{Proof($\subseteq$):}}}
\newcommand{\trans}[3]{$#1:#2\rightarrow{}#3$}
\newcommand{\map}[3]{\text{$\left[#1\right]_{#2}^{#3}$}}
\newcommand{\dime}[1]{\text{dim}(#1)}
\newcommand{\mat}[2]{M_{#1 \times #2}(\R)}
\newcommand{\aug}{\fboxsep=-\fboxrule\!\!\!\fbox{\strut}\!\!\!}
\newcommand{\basecase}{\textsc{\underline{Basis Case:}} }
\newcommand{\itemmnductive}{\textsc{\underline{Inductive Step:}} }
\newcommand{\norm}[1]{\left\lVert#1\right\rVert}
% Call settag{\ldots} first to initialize, and then \para{} for a new paragraph
\newcommand{\para}[1]{\item \tbf{#1}}
\newcommand{\va}{\mathbf{a}}
\newcommand{\vb}{\mathbf{b}}
\newcommand{\vv}{\mathbf{v}}
\newcommand{\vu}{\mathbf{u}}
\newcommand{\vw}{\mathbf{w}}
\newcommand{\vx}{\mathbf{x}}
\newcommand{\ve}{\mathbf{e}}
\newcommand{\vy}{\mathbf{y}}
\newcommand{\vz}{\mathbf{z}}
\newcommand{\vzero}{\mathbf{0}}
% For convenience, I am setting both of these to refer to the same thing.
\newcommand{\ba}{\mathbf{a}}
\newcommand{\bb}{\mathbf{b}}
\newcommand{\bv}{\mathbf{v}}
\newcommand{\bu}{\mathbf{u}}
\newcommand{\bw}{\mathbf{w}}
\newcommand{\bx}{\mathbf{x}}
\newcommand{\be}{\mathbf{e}}
\newcommand{\by}{\mathbf{y}}
\newcommand{\bzero}{\mathbf{0}}
\newcommand{\itemm}[1]{\item \texttt{#1}}


\pdfinfo{
   /Author (Tingfeng Xia)
   /Title  (HTML tag manuel)
   /CreationDate (D:20190531)
   /Subject (Web Development)
}

\title{%
  \textbf{Web Development}\\
  \large Guidance to attributes}
\author{Tingfeng Xia}
\date{2019.05}

\begin{document}


\maketitle
\newpage % make a new page for actual contents, possibly a content table
\mbox{}
\vfill
\noindent by Tingfeng Xia \\


\noindent HyperText Markup Language, commonly referred to as HTML, 
is the standard markup language used to create web pages. Although this langugage itself
is not hard to learn, the huge amount of tags that where involved in the language usually make
new learners feel overwhelmed at first contact. This manuel serves the pure purpose as a 
look up manuel for all the tags. We will proceed alphabetically.
\newpage

\begin{itemize}
	\itemm{<!-- ... -->} is the comment tag, the \texttt{...} inside is the comment
	message. Notice that this supports multiline comments.
	\itemm{<!DOCTYPE>} Ususally we put \texttt{<!DOCTYPE html>} as the first line in a html file
	to tell the browser to use html 5 as the language. Do notice that browsers are indifferent
	to the capitalization, but we do this to follow the convention.
	\itemm{<a>} is the syntax for a hyper-reference, for example we can write 
	\texttt{<a href="/path/to/something">go to something</a>} and this will generate a hyper-linked
	text ``go to something'' which will take you to that something. Similiar to the way it is
	in a Microsoft Office, the hyper-link has color to different its status: blue for not-clicked
	and purple for already accessed. 
	\itemm{<abbr>} defines an abbreviation. You can write, for example, 
	\texttt{<abbr title="university of Toronto">UofT</abbr>}, which will display ``UofT'' on the
	outside and when you hover your mouse over the word, the full name will appear.
	\itemm{<address>} defines contact information. The typical uses (by convention) of the address tag are:
		\begin{enumerate}
			\item If the \texttt{<address>} tag is inside the body element, then it provides the info
			for contacting the owner/writer of the entire document.
			\item If the \texttt{<address>} tag is inside a article element, then it provides the info
			for contacting the owner/writer of that particular article section.
			\item Otherwise, the \texttt{<address>} tag is usually in the ``other'' part of footer 
			section of the page.
		\end{enumerate}
		notice that by very string conventin we never use address tag directly for an address, unless 
		the address is one part of a bigger family of information.
	\itemm{<area>} defines area inside a click-able image, which means area is always inside a map tag.
	We can use area for hyperlinks, for example \texttt{<area shape="rect/circle" coords="???" alt="??" href="?.htm">}, 
	which will direct us to the \texttt{?.htm} refered upon clicked inside the coords for that paticular shape
	\itemm{<article>} defines an area for article. This is paticularly useful for posts in a BBS, articles
	in a blog, news or storys and comments that users leave on a certain site.

\end{itemize}


\end{document}